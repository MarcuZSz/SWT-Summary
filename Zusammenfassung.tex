\documentclass[12pt,a4paper,oneside]{article}
\usepackage[utf8]{inputenc}     
\usepackage[T1]{fontenc}
\usepackage[ngerman]{babel}     
\usepackage{amsmath, amsfonts, amssymb} 
\usepackage{graphicx}           
\usepackage{float}              
\usepackage{caption}  
\usepackage{adjustbox}
\usepackage[hidelinks]{hyperref}
\usepackage[left=2cm,right=2cm,top=2cm,bottom=2cm]{geometry}       

\setcounter{secnumdepth}{5}     
\setcounter{tocdepth}{5}        

\begin{document}

\tableofcontents
\newpage

\section{Softwareprozesse}

\subsection{Vorgehensmodelle}

\subsubsection{Wasserfallmodell}

\begin{figure}[H]

\centering

\begin{minipage}[t]{0.45\textwidth}
\begin{itemize}
\item lineares, plangetriebenes Modell mit aufeinanderbauenden Phasen
\item typische Phasen:
\begin{enumerate}
\item Anforderungsanalyse
\item System- \& Softwareentwurf
\item Implementierung \& Modultests
\item Integration \& Systemtest
\item Betrieb \& Wartung
\end{enumerate}
\item Vorteile: klare Struktur, gut dokumentiert, geeignet für sicherheitskritische Systeme
\item Nachteile: Änderungen sind teuer, spätes Feedback, kaum Flexibilität
\end{itemize}
\end{minipage}
\hfill
\begin{minipage}[t]{0.5\textwidth}
\centering
\vspace{20mm}
\adjustbox{valign=c}{\includegraphics[scale=0.45]{pictures/Wasserfallmodell.png}}
\caption{Wasserfallmodell}
\end{minipage}
\end{figure}

\subsubsection{Inkrementelle Entwicklung}

\begin{itemize}
\item Spezifikation, Entwicklung \& Validierung laufen parallel.
\item Software wird in aufeinander aufbauenden Inkrementen geliefert.
\item Jedes Inkrement liefert nutzbare Funktionalität.
\item Vorteile: Frühe Auslieferung, Kundenfeedback möglich, Anpassungen leichter
\item Nachteile: Mögliche Systemdegradation bei vielen Versionen, Management-Komplexität
\end{itemize}

\begin{figure}[H]
\centering
\includegraphics[scale=0.6]{pictures/inkrementelleImp.png}
\caption{Inkrementelle Entwicklung}
\end{figure}

\subsubsection{Integration \& Konfiguration}

\begin{itemize}
\item Nutzt bestehende Komponenten oder Systeme.
\item Phasen:
\begin{enumerate}
\item Anforderungsspezifikation
\item Komponenten-/Softwareerkennung \& -evaluierung
\item Anforderungsnachbesserung
\item Anpassen \& Integration von Komponenten
\item Konfiguration des Anwendungssystems
\end{enumerate}
\item Vorteile: Schnelle Lieferung, geringere Kosten und Risiken
\item Nachteile: Eingeschränkte Kontrolle, Kompromisse bei Anforderungen
\end{itemize}

\begin{figure}[H]
\centering
\includegraphics[scale=0.75]{pictures/integrationukonfig.png}
\caption{Wiederverwendungsorientiertes Software-Engineering}
\end{figure}

\subsection{Prozessaktivitäten}

\subsubsection{Softwarespezifikation}

\begin{figure}[H]
\begin{minipage}[t]{0.45\textwidth}
\begin{itemize}
\item Definition der Funktionen und Beschränkungen des Systems
\item Ergebnis: Ein Anforderungsdokument, das zwischen Kunde und Entwickler abgestimmt wird
\item Fehler hier sind kritisch, da sie später teuer werden
\end{itemize}
\end{minipage}
\hfill
\begin{minipage}[t]{0.45\textwidth}
\centering
\vspace{-15mm}
\adjustbox{valign=c}{\includegraphics[scale=0.5]{pictures/softspezi.png}}
\caption{Ablauf der Anforderungsanalyse}
\end{minipage}
\end{figure}

\newpage

\subsubsection{Softwareentwurf \& -implementierung}

\begin{itemize}
\item Umsetzung der spezifizierten Anforderungen in ein ausführbares System
\item Umfasst Entwurf (Architektur, Komponenten) und Implementierung (Programmierung)
\end{itemize}

\begin{figure}[H]
\centering
\includegraphics[scale=0.6]{pictures/softent-imp.png}
\caption{allgemeines Modell des Entwurfsprozesses}
\end{figure}

\subsubsection{Softwarevalidierung}
\begin{itemize}
\item Überprüfung, ob das System den Anforderungen und Benutzerbedürfnissen entspricht
\item Aktivitäten: Reviews, Tests, Abnahmen
\item Ziel: Qualitätssicherung und Fehlererkennung vor Auslieferung
\end{itemize}

\begin{figure}[H]
\begin{minipage}[t]{0.5\textwidth}
\centering
\includegraphics[scale=0.6]{pictures/softvali.png}
\caption{Testphasen}
\end{minipage}
\hfill
\begin{minipage}[t]{0.5\textwidth}
\centering
\includegraphics[scale=0.4]{pictures/softvali2.png}
\caption{Testphasen in plangesteuertem Softwareprozess}
\end{minipage}
\end{figure}

\newpage

\subsubsection{Weiterentwicklung von Software}
\begin{itemize}
\item Anpassung bestehender Software an neue Anforderungen, Technologien oder Umgebungen
\item Evolution kontinuierlich \& macht den Unterschied zwischen Entwicklung \& Wartung zunehmend unscharf
\end{itemize}

\begin{figure}[H]
\centering
\includegraphics[scale=0.5]{pictures/evosoft.png}
\caption{Weiterentwicklung eines Softwaresystems}
\end{figure}

\end{document}
